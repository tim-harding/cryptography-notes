\documentclass{article}

\usepackage{amsmath}
\usepackage{amsfonts}
\usepackage[utf8]{inputenc}
\usepackage[english]{babel}
\usepackage{multicol}
\usepackage[margin=0.5in]{geometry}

\setlength{\parindent}{0em}
\setlength{\parskip}{1em}

\newcommand{\curly}[1]{\left\{ #1 \right\}}
\newcommand{\soft}[1]{\left( #1 \right)}
\newcommand{\hard}[1]{\left[ #1 \right]}

\pagenumbering{gobble}

\title{Cryptography Notes}
\author{Tim Harding}

\begin{document}
\begin{multicols*}{2}

\section*{Theorems}

\begin{align*}
    (a\ |\ b) \wedge (b\ |\ c) &\implies a\ |\ c \\
    (a\ |\ b) \wedge (b\ |\ a) &\implies a = \pm b \\
    (a\ |\ b) \wedge (a\ |\ c) &\implies a\ |\ (b \pm c)
\end{align*}
\begin{align*}
    \frac{a}{b} \quad &\longrightarrow  \quad a = bq + r \\
    \gcd(a, b) &= \gcd(b, r) \\
    \gcd(a, b) &= au + bv \\
    au + bv = 1 \quad &\implies \quad \text{$a$ and $b$ are coprime}
\end{align*}
\begin{align*}
    a_1 &\equiv a_2 \pmod{m} \\
    b_1 &\equiv b_2 \pmod{m} \\
    &\Downarrow \\
    a_1 \times b_1 &\equiv a_2 \times b_2 \pmod{m} \\
    a_1 \pm b_1 &\equiv a_2 \pm b_2 \pmod{m} \\
\end{align*}
\begin{align*}
    a \times b \equiv 1 \pmod{m} \quad \Leftrightarrow \quad \gcd(a, m) = 1
\end{align*}
\begin{align*}
    \gcd(a&, m) = 1 \\
    &\Downarrow \\
    \text{$a$ has a multi} & \text{plicative inverse}
\end{align*}
\begin{align*}
    \frac{\mathbb{Z}}{m\mathbb{Z}} &= \curly{0, 1, \ldots, m - 1} \\
    \soft{\frac{\mathbb{Z}}{m\mathbb{Z}}}^* &= \curly{1, 2, \ldots, m - 1}
\end{align*}
\begin{align*}
    x \equiv 1 \pmod{m} \implies m\ |\ (x - 1) \implies xu + mv = 1
\end{align*}
\begin{align*}
    \forall a \in \soft{\frac{\mathbb{Z}}{p\mathbb{Z}}}^*, \exists a^{-1} : a \times a^{-1} \equiv 1 \pmod{p}
\end{align*}

\section*{Algorithms}

\subsection*{Euler}
Calculates $\gcd(a, b)$.
\begin{align*}
    291 &= 252(1) + 39 \\
    252 &= 39(6)  + 18 \\
    39  &= 18(2)  + \boxed{3} \\
    18  &= 3(6)   + 0
\end{align*}

\subsection*{Euler Extended}
Calculates $au + bv = \gcd(a, b)$.
\begin{align*}
    291(1)  + 252(0) &= 291 \\
    291(0)  + 252(1) &= 252 \\
    291(1)  + 252(-1) &= 39 \\
    252(-6) + 39(7)   &= 18 \\
    39(13)  + 18(-15) &= 3 \\
\end{align*}

\end{multicols*}
\pagebreak

\end{document}