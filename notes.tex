\documentclass{article}

\usepackage{fouriernc}
\usepackage[T1]{fontenc}
\usepackage{amsmath}
\usepackage{amssymb}
\usepackage{amsfonts}
\usepackage[utf8]{inputenc}
\usepackage[english]{babel}
\usepackage{multicol}
\usepackage[margin=0.5in]{geometry}

\setlength{\parindent}{0em}
\setlength{\parskip}{1em}

\newcommand{\curly}[1]{\left\{ #1 \right\}}
\newcommand{\soft}[1]{\left( #1 \right)}
\newcommand{\hard}[1]{\left[ #1 \right]}

\pagenumbering{gobble}

\title{Cryptography Notes}
\author{Tim Harding}

\begin{document}
\begin{multicols*}{2}

\section*{Theorems}

\begin{align*}
    (a \mid b) \wedge (b \mid c) &\implies a \mid c \\
    (a \mid b) \wedge (b \mid a) &\implies a = \pm b \\
    (a \mid b) \wedge (a \mid c) &\implies a \mid (b \pm c)
\end{align*}
\begin{align*}
    \frac{a}{b} \quad &\longrightarrow  \quad a = bq + r \\
    \gcd(a, b) &= \gcd(b, r)
\end{align*}
\begin{align*}
    \gcd(a, b) &= au + bv \\
    \gcd(a, b) &= 1 \ \implies \ \text{$a$ and $b$ are coprime}
\end{align*}
\begin{align*}
    a_1 &\equiv a_2 \pmod{m} \\
    b_1 &\equiv b_2 \pmod{m} \\
    &\Downarrow \\
    a_1 \times b_1 &\equiv a_2 \times b_2 \pmod{m} \\
    a_1    \pm b_1 &\equiv a_2    \pm b_2 \pmod{m} \\
\end{align*}
\begin{align*}
    a \times b \equiv 1 \pmod{m} \quad \Leftrightarrow \quad \gcd(a, m) = 1
\end{align*}
\begin{align*}
    \gcd(a&, m) = 1 \\
    &\Downarrow \\
    \text{$a$ has a multi} & \text{plicative inverse}
\end{align*}
\begin{align*}
    \frac{\mathbb{Z}}{m\mathbb{Z}} &= \curly{0, 1, \ldots, m - 1} \\
    \soft{\frac{\mathbb{Z}}{m\mathbb{Z}}}^* &= \curly{1, 2, \ldots, m - 1}
\end{align*}
\begin{align*}
    x \equiv 1 \pmod{m} \implies m \mid (x - 1) \implies xu + mv = 1
\end{align*}
\begin{align*}
    \forall a \in \soft{\frac{\mathbb{Z}}{p\mathbb{Z}}}^*, \exists a^{-1} : a \times a^{-1} \equiv 1 \pmod{p}
\end{align*}
\begin{align*}
    p \nmid a \implies a^{p-1} \equiv 1 \pmod{p} \\
    p  \mid a \implies a^{p-1} \equiv 0 \pmod{p}
\end{align*}
\begin{align*}
    a^{p-1} \equiv \begin{cases}
        1 \pmod{p} & p \nmid a \\
        0 \pmod{p} & p  \mid a
    \end{cases}
\end{align*}
\begin{align*}
    a^{-1} \equiv a^{p-2} \pmod{p}
\end{align*}
\begin{align*}
    \exists g \in \mathbb{F}_p^* , \mathbb{F}_p^* = \curly{g^n : 0 \leq g \leq p - 2}
\end{align*}
\begin{align*}
    p \mid ja \implies p \mid j \vee p \mid a
\end{align*}

$a^{-1}$ can be computed either with the extended Euclidean algorith or by finding $a^{p-2}$ with the fast powering algorithm.

\subsection*{Prime number theorem}
\begin{align*}
    \lim_{x \rightarrow \infty} \frac{\pi(x)}{x / \ln(x)} &= 1 \\
    \pi(x) &\approx \frac{x}{\ln x}
\end{align*}

\subsection*{RSA}
The basis of RSA is solving the hard problem
\begin{align*}
    x^e \equiv c \pmod{N}
\end{align*}

When $N$ is prime, then compute $de \equiv 1 \pmod{p-1}$. The difficulty comes from finding $d$ when $N = pq$ and must be factored in order to solve the congruence.

\subsubsection*{Procedure}
Bob picks large, secret primes $p$ and $q$ and an exponent $e$. He solves for $d$ in $ed \equiv 1 \pmod{(p-1)(q-1)}$ and holds on to it. He publishes $N = pq$ and $e$.

Alice encrypts a message $m$ to send to Bob by computing $c \equiv m^e \pmod{N}$.

Bob decrypts the message using $m \equiv c^d \pmod{N}$.

\end{multicols*}

\section*{Algorithms}

\subsection*{Euclidean}
Calculate $\gcd(291, 252)$.
\begin{align*}
    291 &= 252(1) + 39        \\
    252 &= 39(6)  + 18        \\
    39  &= 18(2)  + \boxed{3} \\
    18  &= 3(6)   + 0
\end{align*}

\subsection*{Extended Euclidean}
Calculate $291u + 252v = \gcd(291, 252)$.
\begin{align*}
    291(1)  + 252(0)   &= 291 \\
    291(0)  + 252(1)   &= 252 \\
    291(1)  + 252(-1)  &= 39  \\
    291(-6) + 252(7)   &= 18  \\
    291(13) + 252(-15) &= 3
\end{align*}
$(u, v) = (13, -15)$

$291 \times 13 + 252 \times -15 = 3$

\subsection*{Fast powering}
Calculate $17^{811} \pmod{643}$
\begin{align*}
    811_{10} = 1100101011_2
\end{align*}
\begin{align*}
    17^2 &\equiv 289 \pmod{643} \\
    289^2 &\equiv 574 \pmod{643} \\
    574^2 &\equiv 260 \pmod{643} \\
    & \qquad \quad \vdots
\end{align*}
\begin{align*}
    17^{1} \times 289^{1} \times 574^{0} \times 260^{1} \times 85^{0} \times 152^{1} \times 599^{0} \times 7^{0} \times 49^{1} \times 472^{1}
\end{align*}
\begin{align*}
    17^{811} \equiv 621 \pmod{643}
\end{align*}

\subsection*{Pollard $p-1$ factorization}
\begin{enumerate}
    \item Let $a = 2$ (for example)
    \item For $1 < i < \text{bound}$
    \begin{enumerate}
        \item $a = a^i$
        \item $d = \gcd(a - 1, N)$
        \item If $1 < d < N$, return $d$
    \end{enumerate}
    \item Try another $a$
\end{enumerate}

Example:
\begin{align*}
    \vdots \\
    2^{12!} - 1 &\equiv 6680550 \pmod{13927189} & \gcd(2^{12!} - 1, 13927189) &= 1 \\
    2^{13!} - 1 &\equiv 6161077 \pmod{13927189} & \gcd(2^{13!} - 1, 13927189) &= 1 \\
    2^{14!} - 1 &\equiv 879290  \pmod{13927189} & \gcd(2^{14!} - 1, 13927189) &= 3823 \\
\end{align*}
$3643 \times 3823 = 13927189$

\subsection*{Miller-Rabin}
Testing $n$ with witness $a$:
\begin{enumerate}
    \item If $n$ is even or $1 < \gcd(a, n) < n$ then return composite
    \item $n - 1 = 2^k q$ such that $q$ is odd
    \item $a = a^q$
    \item If $a \equiv 1 \pmod{n}$ then the test fails
    \item for $i$ in $0...k-1$
    \begin{enumerate}
        \item If $a \equiv -1 \pmod{n}$ then the test fails
        \item $a = a^2 \pmod{n}$
    \end{enumerate}
    \item Return composite
\end{enumerate}

Example:
$n = 561$ and $n - 1 = 560 = 2^4 \times 35$
\begin{align*}
    2^{35} \equiv 263 \pmod{561} \\
    2^{2 \times 35} \equiv 263^2 \equiv 166 \pmod{561} \\
    2^{4 \times 35} \equiv 166^2 \equiv 67 \pmod{561} \\
    2^{8 \times 35} \equiv 67^2 \equiv 1 \pmod{561} \\
\end{align*}
561 is composite.

\subsection*{Trial division}
Divide $n$ by all primes up to $\sqrt{n}$. If a prime $p$ divides $n$, let $n = \frac{n}{p}$, keep $p$ as a factor, and keep $p$ in case it divides $n$ multiple times.

\end{document}