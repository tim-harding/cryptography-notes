\documentclass{article}

\usepackage{fouriernc}
\usepackage[T1]{fontenc}
\usepackage{amsmath}
\usepackage{amssymb}
\usepackage{amsfonts}
\usepackage[utf8]{inputenc}
\usepackage[english]{babel}
\usepackage{multicol}
\usepackage[margin=0.5in]{geometry}

\setlength{\parindent}{0em}
\setlength{\parskip}{1em}

\newcommand{\curly}[1]{\left\{ #1 \right\}}
\newcommand{\soft}[1]{\left( #1 \right)}
\newcommand{\hard}[1]{\left[ #1 \right]}

\pagenumbering{gobble}

\title{Cryptography Notes}
\author{Tim Harding}

\begin{document}
\begin{multicols*}{2}

\section*{Theorems}

\begin{align*}
    (a \mid b) \wedge (b \mid c) &\implies a \mid c \\
    (a \mid b) \wedge (b \mid a) &\implies a = \pm b \\
    (a \mid b) \wedge (a \mid c) &\implies a \mid (b \pm c)
\end{align*}
\begin{align*}
    \frac{a}{b} \quad &\longrightarrow  \quad a = bq + r \\
    \gcd(a, b) &= \gcd(b, r)
\end{align*}
\begin{align*}
    \gcd(a, b) &= au + bv \\
    \gcd(a, b) &= 1 \ \implies \ \text{$a$ and $b$ are coprime}
\end{align*}
\begin{align*}
    a_1 &\equiv a_2 \pmod{m} \\
    b_1 &\equiv b_2 \pmod{m} \\
    &\Downarrow \\
    a_1 \times b_1 &\equiv a_2 \times b_2 \pmod{m} \\
    a_1    \pm b_1 &\equiv a_2    \pm b_2 \pmod{m} \\
\end{align*}
\begin{align*}
    a \times b \equiv 1 \pmod{m} \quad \Leftrightarrow \quad \gcd(a, m) = 1
\end{align*}
\begin{align*}
    \gcd(a&, m) = 1 \\
    &\Downarrow \\
    \text{$a$ has a multi} & \text{plicative inverse}
\end{align*}
\begin{align*}
    \frac{\mathbb{Z}}{m\mathbb{Z}} &= \curly{0, 1, \ldots, m - 1} \\
    \soft{\frac{\mathbb{Z}}{m\mathbb{Z}}}^* &= \curly{1, 2, \ldots, m - 1}
\end{align*}
\begin{align*}
    x \equiv 1 \pmod{m} \implies m \mid (x - 1) \implies xu + mv = 1
\end{align*}
\begin{align*}
    \forall a \in \soft{\frac{\mathbb{Z}}{p\mathbb{Z}}}^*, \exists a^{-1} : a \times a^{-1} \equiv 1 \pmod{p}
\end{align*}
\begin{align*}
    p \nmid a \implies a^{p-1} \equiv 1 \pmod{p} \\
    p  \mid a \implies a^{p-1} \equiv 0 \pmod{p}
\end{align*}
\begin{align*}
    a^{p-1} \equiv \begin{cases}
        1 \pmod{p} & p \nmid a \\
        0 \pmod{p} & p  \mid a
    \end{cases}
\end{align*}
\begin{align*}
    a^{-1} \equiv a^{p-2} \pmod{p}
\end{align*}

\section*{Algorithms}

\subsection*{Euler}
Calculate $\gcd(291, 252)$.
\begin{align*}
    291 &= 252(1) + 39        \\
    252 &= 39(6)  + 18        \\
    39  &= 18(2)  + \boxed{3} \\
    18  &= 3(6)   + 0
\end{align*}

\subsection*{Euler Extended}
Calculate $291u + 252v = \gcd(291, 252)$.
\begin{align*}
    291(1)  + 252(0)  &= 291 \\
    291(0)  + 252(1)  &= 252 \\
    291(1)  + 252(-1) &= 39  \\
    252(-6) + 39(7)   &= 18  \\
    39(13)  + 18(-15) &= 3
\end{align*}
$(u, v) = (13, -15)$

$291 \times 13 + 252 \times -15 = 3$

\subsection*{Fast powering}
Calculate $17^{811} \pmod{643}$
\begin{align*}
    811_{10} = 1100101011_2
\end{align*}
\begin{align*}
    17^2 &\equiv 289 \pmod{643} \\
    289^2 &\equiv 574 \pmod{643} \\
    574^2 &\equiv 260 \pmod{643} \\
    & \qquad \quad \vdots
\end{align*}
\begin{align*}
    17^{1} \times 289^{1} \times 574^{0} \times 260^{1} \times 85^{0} \times 152^{1} \times 599^{0} \times 7^{0} \times 49^{1} \times 472^{1}
\end{align*}
\begin{align*}
    17^{811} \equiv 621 \pmod{643}
\end{align*}

\end{multicols*}
\pagebreak

\end{document}